\documentclass[reqno,openany,12pt]{amsbook}
\usepackage{amsmath}

%  openany option dumps the blank pages between chapters when the next
%  chapter starts on odd page number; following command had similar
%  effect
%  \let\cleardoublepage\clearpage
%  NB need an abstract or get a blank page between title and contents


\renewcommand{\baselinestretch}{1.35}


%  ************  begin my definitions  *******************

% *** theorem etc. commands ***

\newtheorem{thm}{Theorem}%[section]
\newtheorem{lemma}[thm]{Lemma}
\newtheorem{cor}[thm]{Corollary}

\theoremstyle{definition}
\newtheorem{definition}[thm]{Definition}
\newtheorem*{ass}{Assumption S}

\theoremstyle{remark}
\newtheorem{remark}[thm]{Remark}

%\numberwithin{equation}{section}

\newenvironment{mylist}{\begin{enumerate}
\def\labelenumi{\theenumi}
\renewcommand{\theenumi}{(\roman{enumi})}
}{\end{enumerate}}

% *** greek commands ***

\newcommand\al{\alpha}
\newcommand\be{\beta}
\newcommand\ga{\gamma}
\newcommand\Ga{\Gamma}
\newcommand\de{\delta}
\newcommand\De{\Delta}
\newcommand\ep{\epsilon}
\newcommand\ka{\kappa}
\newcommand\la{\lambda}
\newcommand\La{\Lambda}
\newcommand\om{\omega}
\newcommand\Om{\Omega}
\newcommand\si{\sigma}
\newcommand\Si{\Sigma}
\renewcommand\th{\theta}
\newcommand\Th{\Theta}

% *** tilde/bar/bold commands ***


\newcommand\bg{\bar g}
\newcommand\bh{\bar h}
\newcommand\bu{\bar u}

\newcommand\bq{\boldsymbol{q}}
\newcommand\br{\boldsymbol{r}}
\newcommand\bv{\boldsymbol{v}}

% *** Bbb commands ***

\newcommand\Q{\mathbb{Q}}
\newcommand\R{\mathbb{R}}
\newcommand\Nf{\mathbb{N}}
\newcommand\Zf{\mathbb{Z}}

% *** script commands ***

\newcommand\I{{\mathcal I}}

\newcommand\B{{\mathcal B}}

% *** brackets commands ***

\newcommand\lan{\langle}
\newcommand\ran{\rangle}

% *** various maths commands ***

\newcommand\X{\times}
\newcommand{\tow}{\rightharpoonup}
\newcommand{\pa}{\partial}
\newcommand\rot{{\rm Rot}}


%  ************  end my definitions  *******************


\begin{document}


\title{Title}
\author{Author\\
{\small
MSc Dissertation\\[-1 ex]
Department of Mathematics and Computer Science\\[-1 ex]
Heriot Watt University\\[-1 ex]
date
}
}
\bigskip

\author{Supervised by Prof Bryan Rynne}


\begin{abstract}
This project will investigate some aspects of rational approximation to
real numbers.
It is well known that any irrational number can be estimated arbitrarily
close by rationals, i.e. for any $\ep>0$ and irrational $x \in \R$,
there exists integers $p$,$q$ such that
\begin{equation}  \label{abs_basic_app.eq}
\left| {x-\frac{p}{q}} \right| < \ep
\end{equation}
This is known as Diophantine approximation, named after Diophantus of
Alexandria.

\end{abstract}


\maketitle


 \setcounter{page}{0}


\tableofcontents


\chapter{Introduction}

This project will investigate some aspects of rational approximation to
real numbers.
It is well known that any irrational number can be estimated arbitrarily
close by rationals, i.e. for any $\ep>0$ and irrational $x \in \R$,
there exists integers $p$,$q$ such that
\begin{equation}\label{firsteq}
\left| {x-\frac{p}{q}} \right| < \ep
\end{equation}
This is known as Diophantine approximation, named after Diophantus of Alexandria.
A good example to illustrate this is the rational approximation to the
value of $\pi$, which is an irrational number equal to
$3.141592653589793\dots$.
Below are some early approximations:-
\bigskip
\begin{itemize}
\item
The Rhind Papyrus ($\approx 1650$ B.C.):
$\pi \approx 4(8/9)^2= 3.16\dots,$
\item
Old Testament ($\approx 1000$ B.C.): $\pi \approx 3$,
\item
Archimedes (287-212 B.C.): $\pi \approx \frac{22}{7} = 3.142\dots,$
\item
Tsu Ching Chi ($\approx 500$ A.D.):
$\pi \approx \frac{355}{133} = 3.1415929\dots.$
\end{itemize}
\bigskip



Of course, we can obtain
`better' approximations to the value of $x$ by using larger denominators
$q$.
Hence, an obvious question to ask is: `can we make the dependence of the
level of approximation on the size of $q$ more precise?',
that is,
`can we express the inequality in (1) in a more quantitative way?'.
If we replace $\ep$ in (1) by $1/{q^\al}$,
for some $\al > 0$,
then we have the inequality
\begin{equation} \label{first_approx.eq}
\left| {x-\frac{p}{q}}\right| <\frac{1}{q^\al}
\end{equation}
which we want to hold for arbitrarily large $q$,
that is, we want (2) to hold infinitely often.
Thus, we  investigate the set of $x$ and $\al$ for which the
inequality
(2) holds for infinitely many integers $q$.

This basic approximation of a single number $x$ by rationals can be
extended and generalised in various ways, some of which will be
described in this project.
A good introduction to the subject is in \cite{Cassels}.

\chapter{Basic Definitions and Notation}


\section{Lebesgue measure zero}


\begin{definition}
If $A \subset \R^k$ then
$$
|A| := \sup \{ |x-y| : x,\,y \in \bar A \}.
$$
If $|A| < \infty$ then $A$ is {\em bounded};
otherwise it is {\em unbounded}.
\end{definition}


\begin{definition}
A set $X \subset \R^k$ has {\em $($Lebesgue$)$ measure zero}, if for any
$\ep>0$ there exists a set of intervals $I_1,I_2,I_3,\dots$ such that
$$X\subset I_1\cup I_2\cup I_3,\dots$$
and
$$\sum_{i=1}^\infty |I_i|<\ep.$$
\end{definition}

Below are some examples of sets with Lebesgue measure zero.
\begin{enumerate}
\item
$X_1 =\lbrace x \rbrace,$
\item
$X_2 =\lbrace x_{1},\dots,x_{n}\rbrace,$
\item
$X_3 =\lbrace x_{1},x_{2},\dots \rbrace$.
\end{enumerate}

For example, to prove that $X_3$ has measure zero, let $\ep>0$ and
define
$$
I_i = (x_i - \frac14 \ep 2^{-i}, x_i + \frac14 \ep 2^{-i}), \quad i=1,2,\dots .
$$
Then for each $i=1,2,\dots,$
$
|I_{i}| = \frac12 \ep 2^{-i},
$
and hence
$$
\sum_{i=1}^\infty |I_{i}| = \frac12 \ep \left(\frac12 + \frac14 +
\frac18 +\dots \right) = \frac12 \ep < \ep .
$$

\begin{definition}
A set $X \subset \R^k$ has  {\em full} (Lebesgue) measure if the
complement $X^c := \R^k  \setminus X$ has  measure zero.
\end{definition}

Throughout this project we will refer to Lebesgue measure zero or
full Lebesgue measure simply as measure zero or full measure.


\section{Hausdorff dimension}

In this section we briefly describe the definition of the
Hausdorff dimension of an arbitrary set $F \subset \R$.


\begin{definition}
Let $F \subset \R$ be an arbitrary non-empty set.
A finite or countably infinite collection of
sets $\{ U_i \} = \{ U_1,U_2,\dots\}$ in $\R$ is a {\em cover} of $F$ if
$$
F \subset U_1 \cup U_2 \cup \dots
$$
(in general, one could also allow uncountable collections of sets as covers,
but for our purposes here we won't consider such coverings).

If, in addition,  $|U_i| \le \de$ for all $i$ then $\{ U_i \}$ is a {\em $\de$-cover}
of $F$.
\end{definition}

For any $s \ge 0$ and $\de >0$ we define
\begin{equation}  \label{hausmeasde.eq}
H_\de^s(F) := \inf \Bigl\{ \sum_i |U_i|^s :
\text{$\{U_i\}$ is a $\de$-cover of $F$ }\Bigr\} .
\end{equation}
NB. the summation in \eqref{hausmeasde.eq}, and below, may be finite or
countably infinite depending on whether the covering $\{U_i\}$ is finite
or countable.


\begin{thebibliography}{99}

\bibitem{Bovey}
J. D. Bovey, M. M. Dodson,
The Hausdorff dimension of systems of linear forms
{\em Acta Arithmetica}
(1986) 337-358.

\bibitem{Cassels}
J. W. S. Cassels,
{\em An Introduction to Diophantine Approximation},
Cambridge University Press, 1965.



\end{thebibliography}


\end{document}
