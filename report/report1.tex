% http://tex.stackexchange.com/questions/192817/two-sided-amsbook-with-custom-title-page
% for reference
% \documentclass[twoside, reqno, openright ,12pt]{amsbook}
 

\documentclass[reqno,openany,12pt]{amsbook}
%  openany option dumps the blank pages between chapters when the next
%  chapter starts on odd page number; following command had similar
%  effect
%  \let\cleardoublepage\clearpage
%  NB need an abstract or get a blank page between title and contents

\usepackage{amsmath}

% \renewcommand{\baselinestretch}{1.35}
% changed from:
% http://tex.stackexchange.com/a/79155
\usepackage{setspace}
\setstretch{1.5}


%  ************  begin my definitions  *******************

% *** theorem etc. commands ***

\newtheorem{thm}{Theorem}%[section]
\newtheorem{lemma}[thm]{Lemma}
\newtheorem{cor}[thm]{Corollary}

\theoremstyle{definition}
\newtheorem{definition}[thm]{Definition}
\newtheorem*{ass}{Assumption S}

\theoremstyle{remark}
\newtheorem{remark}[thm]{Remark}

%\numberwithin{equation}{section}

\newenvironment{mylist}{\begin{enumerate}
\def\labelenumi{\theenumi}
\renewcommand{\theenumi}{(\roman{enumi})}
}{\end{enumerate}}

% *** greek commands ***

\newcommand\al{\alpha}
\newcommand\be{\beta}
\newcommand\ga{\gamma}
\newcommand\Ga{\Gamma}
\newcommand\de{\delta}
\newcommand\De{\Delta}
\newcommand\ep{\epsilon}
\newcommand\ka{\kappa}
\newcommand\la{\lambda}
\newcommand\La{\Lambda}
\newcommand\om{\omega}
\newcommand\Om{\Omega}
\newcommand\si{\sigma}
\newcommand\Si{\Sigma}
\renewcommand\th{\theta}
\newcommand\Th{\Theta}

% *** tilde/bar/bold commands ***


\newcommand\bg{\bar g}
\newcommand\bh{\bar h}
\newcommand\bu{\bar u}

\newcommand\bq{\boldsymbol{q}}
\newcommand\br{\boldsymbol{r}}
\newcommand\bv{\boldsymbol{v}}

% *** Bbb commands ***

\newcommand\Q{\mathbb{Q}}
\newcommand\R{\mathbb{R}}
\newcommand\Nf{\mathbb{N}}
\newcommand\Zf{\mathbb{Z}}

% *** script commands ***

\newcommand\I{{\mathcal I}}

\newcommand\B{{\mathcal B}}

% *** brackets commands ***

\newcommand\lan{\langle}
\newcommand\ran{\rangle}

% *** various maths commands ***

\newcommand\X{\times}
\newcommand{\tow}{\rightharpoonup}
\newcommand{\pa}{\partial}
\newcommand\rot{{\rm Rot}}


%  ************  end my definitions  *******************


% title info
% fix me later

\title{Intrusion Detection using Outliers in a modern Cyber Security Dataset}
\author{Nikolaos Perrakis\\
{\small
MSc Project\\[-1 ex]
University of Southampton\\[-1 ex]
Faculty of Physical Sciences and Engineering\\[-1 ex]
Electronics and Computer Science\\[-1 ex]
}
}


\begin{document}


\maketitle

\frontmatter


\chapter*{Abstract}
\setcounter{page}{1}
This is the summary of the project describing the details of the analysis on the ADFA-LD 12 dataset.



\chapter*{Acknowledgements}

Thank you!!


\tableofcontents
\listoffigures


\mainmatter

\chapter{Introduction}

In the industry we are currently experiencing what many people call the fourth industrial revolution. The main characteristics of this disruptive process are:
\begin{itemize}
\item The ubuquitous pressence of connected devices, collectively called Internet of Everything.
\item The ability of cyber system to interact with and affect the physical world creating the so called cyber-physical systems (CPS).
\item The growth of distributed computing and the capabilities that it allows.
\item The evolution of Machine Learning enabling cyber-physical systems with increased autonomy.\footnote{A good example of an autonomous cyber-physical system are self-driving cars.}
\end{itemize}
One common denominator of the aspects of the fourth industrial revolution is increased connectivity of computing devices. This brings the negative side effect that more facets of our life and society are in exposed online making cyber-security even more important. During the last couple of years we have seen many examples of this danger. 

Cyber security generally consists of two parts. Intrusion Prevention Systems (IPS) and Intrusion Detection Systems (IDS). Put it simply an IPS prevents the attacker from getting in and an IDS detects him once he is in.


\chapter{Literature Review}

Review me!



\begin{thebibliography}{99}


\bibitem{dat2} Gideon Creech, Jiankun Huy \emph{Generation of a new IDS Test Dataset: Time to Retire the KDD Collection}, 
2013 IEEE Wireless Communications and Networking Conference (WCNC)


%\bibitem{Bovey}
%J. D. Bovey, M. M. Dodson,
%The Hausdorff dimension of systems of linear forms
%{\em Acta Arithmetica}
%(1986) 337-358.
%
%\bibitem{Cassels}
%J. W. S. Cassels,
%{\em An Introduction to Diophantine Approximation},
%Cambridge University Press, 1965.



\end{thebibliography}


\end{document}
