\documentclass{beamer}

\usepackage{beamerthemesplit}

\title{Intrusion Detection using Outliers in a Cybersecurity Dataset}
\author{Nikolaos Perrakis}
\date{\today}

\begin{document}

\frame{\titlepage}

\section[Outline]{}
\frame{\tableofcontents}

\section{Introduction}
%\subsection{Overview of the Beamer Class}
\frame
{
  %\frametitle{Introduction}

  \begin{itemize}
  \item Anomaly and Outlier Detection
  \item 4\textsuperscript{th} Industrial Revolution
  \item Intrusion Detection Systems
  \end{itemize}
}

\section{Datasets}
\subsection{DARPA/KDD Cup Dataset}
\frame
{
DARPA 1998 and KDD Cup 1999 Intrusion Detection Datasets are the first well known attempts to create a solid IDS dataset. However they have many problems\cite{ids3}:
\begin{itemize}
\item Generation procedure could have been more realistic.
\item Software used is outdated and misrepresentative of current IT landscape.
\item Artefacts of the simulation cause overestimation of efficiency.
\item Inconsistency in labels and attacks used.
\end{itemize}
}

\subsection{Modern Datasets}
\frame
{
The main dataset we will use during the MSc Project:\\
\textbf{ADFA LD12}\cite{dat2} Dataset:
\begin{itemize}
\item Representative of modern attack structure and methodology.
\item Updated software with the inclusion of a component with a known vulnurenability.
\end{itemize}
Another good candidate is the \textbf{AWID 2015}\cite{dat3} Dataset.\\ 
10 Gb in size, we can check scalability of our methods.
}

\section{Feature Engineering}
\frame
{Our Data:\\
A (time) series of kernel system calls!

}


\section{Machine Learning Algorithms}
\frame
{
We will focus on Statistical Based Outlier Detection.
\begin{itemize}
\item Gaussian Mixture Model Anomaly Detection\cite{out2}
\item Support Vector Machine for Novelty Detection\cite{out3}
\item Outlier Rejecting Regression\cite{out6}
\end{itemize}
}

\section{Research Prospects}
\frame{
\begin{itemize}
\item Apply Machine Learning Anomaly Detection Algorithms in Intrusion Detection Settings.
\item Improve Efficiency of Anomaly Detection Algorithms in Intrusion Detection Settings.
\item Improve Scalability of algorithms for applications in modern IT infrastructure.
\end{itemize}
}

\section{Bibliography}
\frame
{
\scriptsize
\bibliographystyle{acm}
\begin{thebibliography}{6}
\bibitem{out2} S. Roberts and L. Tarassenko, \emph{A probabilistic resource allocating network for novelty detection}, Neural Computation, vol. 6, no. 2, pp. 270 - 284, 1994.
\bibitem{out3} P. Hayton, B. Sch¨olkopf, L. Tarassenko, and P. Anuzis, \emph{Support vector novelty detection applied to jet engine vibration spectra}, in NIPS, pp. 946–952, 2000.
\bibitem{out6} Y. Gunawardana, S. Fujiwara, A. Takeda, J. Woo, C. Woelk, and M. Niranjan, \emph{Outlier detection at the transcriptomeproteome interface}, Bioinformatics, 2015.

\bibitem{ids3} Mohiuddin Ahmed, Abdun Naser Mahmood, Jiankun Hu \emph{A survey of network anomaly detection techniques}, Journal of Network and Computer Applications. Vol. 60, January 2016, p. 19-31

\bibitem{dat2} Gideon Creech, Jiankun Huy \emph{Generation of a new IDS Test Dataset: Time to Retire the KDD Collection}, 
2013 IEEE Wireless Communications and Networking Conference (WCNC)
\bibitem{dat3} Constantinos Kolias, Georgios Kambourakis, Angelos Stavrou, and Stefanos Gritzalis (2015) \emph{Intrusion Detection in 802.11 Networks: Empirical Evaluation of Threats and a Public Dataset} IEEE Communication Surveys \& Tutorials, Vol. 18, No. 1, 2016
\end{thebibliography}
}
\end{document}
